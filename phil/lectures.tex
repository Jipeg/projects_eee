\documentclass[a4paper,12pt]{article}

\usepackage{cmap}					% поиск в PDF
\usepackage[T2A]{fontenc}			% кодировка
\usepackage[utf8]{inputenc}			% кодировка исходного текста
\usepackage[english,russian]{babel}	% локализация и переносы

\author{ничего не забыл и лекций ни одной не пропустил Николай Андронов}
\title{Конспект лекций Милославова А.С.}
\date{\today}

\begin{document}
\maketitle
\tableofcontents
%%%%%%%%%%%%%%%%%%%%%%%%%%%%%%%%%%%%%%%%%%%%%%%%%%%%%%%%%%%%%%%
\section{Лекция 1. 7 сентября}
Ваш бюст и походка напомнили мне античность. Философия --- любовь к мудрости. Физики изучают нечто изменчивое. Главное в природе --- движение. Матан --- вечное, неизменное, существующие благодаря существуемым???.\\
3 типа мировоззрения: миф, религия, философия.

1. Миф: синкритичное т.е. слитное. этика+эстетика --- всё соединено. Лосев "Диалектика мифа" (1920е) эпизод из конца 19-го века. \\
Троянская война. 3 богини: Афродита, Гера и Афина решили выяснить, кто самая красивая. (боги такие же как люди, с теми же свойствами). Они выбрали Париса, чтобы он рассудил их. Афродита предложила Елену, жену Минелая. Минелай и все цари чувствовали коллективное унижение. Про Агамемнома и его дочь. Кризис родоплеменных отношений.\\
Итог: синкритизм, эмоции и чувства над разумом, коллективное над индивидуальным

2. Религия: удвоение Мира. Физиология верующего. Процедура обучения молитвенным слезам. Блаженная Ксения

3. Философия. Разум\\
Почему в греции появилась философия и наука? Комплекс причин:

1. Крито-мекенская культура, далее архейцы, далее классическая греция (гомер, полисы и т.д., греки понимали прошлое). ОДновременно: Спарта --- монархия, Афины --- демократия. \\
а) в греции царь = первый среди равных, а не бог --- фараон, жреческие функции\\
б) демократия: значительная часть населения участвует в управлении (но не все)\\
в) тиран --- правитель, не наследник\\
Есть способы разные => надо сравнивать => надо определить: справедливость, благо и т.д. - ?\\
Платон.государство: справедливое государство\\
Печаль древнего грека = изгнание из города => философия

2. Мореплавание, греки впитывают в себе моменты других культур, фалес привез геометрию из Египта\\
первые школы: не Афины! Милетус, Эфес, пифагорийцы еще где-то ??? \\
--- всё это след диалога культур.\\
корабельная ситуация: знания капитана определяло судьбу тех, кто на корабле\\
логос=слово капитана =(софисты) "великий властитель" = огонь (как стихия)

3. Древнегреческая религия.\\
а) не знали священных книг, текстов\\
б) олимпийские боги (официальная религия в классике) не являлись образцами в морально-этическом плане, их боялись, задобряли, но не уважали; Афина = псина, муха (с) Арес (Иллиада)\\
Ванхические девичники, Ванх --- Дионис.\\
в) вопрос о бессметрии души (отличие от Египта)\\
в египте: жизнь на земле --- подготовка к "аиду"\\
греки: после смерти --- ущербное состояние\\
Одиссея: попадает Одиссей в Аид и встречает душу Ахилеса, была назначена царем мертвых. "Я б лучше был батраком на земле, чем тут царем среди мертвых"\\
г) боги = очеловеченные силы природы\\
д) герои = сын бога и смертной женщины \\
смысл героизма = противостояние судьбе, року\\
принцип фатализма: рок, судьба диктуют как жить\\
греческий герой: судьба не мешает мне преследовать мои планы\\
е) child free и ЛГБТ\\
Эдип убил даже не зная что это был его pap\'a. А в городе чудище, треба тянок. Убил Эдип чудище, Эдипа царем сробили, женился не зная на своей mam\'a. Когда всё вскрылось, то все сдохли. Он пытался, он герой!\\
ж) Античная философия --- попытка понять гармонию и красоту космоса. Лосев "История эстетики". Никакого бога вне мироздания! Есть лишь космос, который был всегда.
%%%%%%%%%%%%%%%%%%%%%%%%%%%%%%%%%%%%%%%%%%%%%%%%%%%%%%%%%%%%%%%
\section{Лекция 2. 14 сентября. Пропущена} пропустил
\section{Лекция 3. 21 сентября. Пропущена} болел
\section{Лекция 4. 28 сентября}
Мы остановились на Иллиадах.

Демокрит. Античный атомизм. Атом --- бытие Иллиады. Всё сущее = атомы, перемещающиеся в пустоте. Демокрита не изгнали, а наградили за его трактат "О природе". Парменид и другие от Фалеса до Демокрита --- всё "о природе и космосе". Дальше --- общество и человек!

Софист --- это эксперт, платный учитель красноречия. Они стали публично учить людей, просвещать. Именно они совершили переворот. Пример софиста --- Протагор: человек --- это мера всех вещей. Какой ветер: cold или hot? Всегда есть 2 мнения, а человек выбирает => человек --- мера. Предшественники экзистенциализма. Различают законы природы и законы общества. Нарушение природы => не ущерб; нарушение общества => ущерб, если поймают!

Сократ. знаем от Платона и Ксенофонта (не мог ничего придумать, но писал, когда сам был old) --- 2 разных Сократа. Сократ считал, что человек носит уже истину внутри себя и задача философа --- помочь людям родить мысль (как окушерка). Метод маевтики (наводящие вопросы). Но, чтобы мы захотели родить истину, надо разозлить людей. Жалил сограждан своими вопросами как овод скотину!\\
Апологея Сократа. Федон.\\
1. добродетели $\approx$ счастье, добро $\approx$ знание. Люди в тисках незнания. Человек = разумная душа. Сократ рулил евпропеоидной философией до середины 19го века. А потом стали работать гильятиной: кто мешает общественному балгу --- того в минус. Битва при Галиполе

Платон. Кодр спас город. Жизнь пришла в упадок! демократия --- плохо, така как убили Сократа. О Платоне --- Лосев "Плотон". \\
Как устроить государство?\\
1) теория идей; 2) бессмертие души; 3) знание как припоминание; 4) учение о государстве; 5) космология Алптона

1. Идея --- это не мысль, а то по поводу чего мысль думает! Мир идей --- над небом. Идея единого: всё $\exists$ в силу единства. Первый теоретик тоталитарного государства. Признаки тоталитарности:
1) $\exists$ группа лиц, претендующие на владение истины, управляют\\
2) ограничение на частную или личную собственность\\
2) вмешательство в личную жизнь государством\\
4) система психологического образования (пионеры)\\
Все это было у Платона. Вы --- для государства!

2. Учение о душе. Что есть в нас, чтобы познать мир идей? Органы чувств часто приводят нас в заблуждение, не подходит. Болезни не подходят. Войны ради денег, а они только для заботы о теле, тело --- "плохо"\\
=> $\exists$ душа --- бессмертная, нематериальная так как мир идей такой. 4 аргумента души. Душа --- повозка с двумя лошадьми: повозка=разум, лошади=аффектно-инстинктивная и вожделеющая части души\\
если разум справляется, то душа --- в мир идей; каждой части души --- добродетель:\\
аффективная --- умеренность --- люди (простолюдины)\\
вожделеющая --- мужество --- стражи\\
разум --- мудрость --- философы\\
Так вот ответ: душа уже была в Мире идей. Познание = припоминание.
%%%%%%%%%%%%%%%%%%%%%%%%%%%%%%%%%%%%%%%%%%%%%%%%%%%%%%%%%%%%%%%%%%%%
\section{Лекция 5. 5 октября}
Платон -> Сиракузы, воспитывает сына Диониссия I (тирана Сиракуз). Потом его продал в рабство сам этот Диониссий.

Аристотель. ученик Платона. Гай Свитоний Траквил "В жизни 12 цезаря". Фильм --- всё по тексту. Калигула. Сериал "Рим" НВО. \\
Период эллинизма и римской империи: \\
"Как человеку стать счастливым в Мире, в котором от него ничего не зависит?"

1. Киники (Циники).\\
Ученик Сократа Антисфен (435-360)\\
Самый известный --- Диоген Синопский. картины Александр Мак. и Диоген. \\
1358-1968 (годы жизни, означают 1358 по исламскому календарю, 1968 --- грегорианский)\\
Ататюрк в 1920-е годы совершил модернизацию Турции. Для турок --- это еще круче, чем Петр I.\\
Принцип философии киников: цель жизни --- счастье (Антисфен), мешают: страх, бессмысленная погоня за странными ценностями. Слава и общие ценности --- ненужные вещи. "Пусть дети врагов ваших живут в роскоши". Для счастья надо:\\
1. Аскезис --- метод и путь к свободе и добродетели; отказ от материальных благ, тренировка души и тела до готовности противостоять неугодам. "если у вас нет собаки, то ее не отравит сосед". Аскезис приводит к\\
2. Автаркия: самодостаточность\\
3. Апальдэусия --- упрощение жизни (киник = собачий)\\
Еще есть Диаген Лаэртский "О жизни учении мудрецов"\\
А Синопский --- провокатор, как Сократ. Но у Сократа --- интеллектуальные иронии, а Диаген --- хам. Диаген бегал с фонарем: "ищу человека".

2. Стоики. Зенон (333-269)\\
2--1 века --- средняя стоя\\
1 век--180 год --- поздняя стоя\\
последний стоик --- Марк Аврелий (121-180гг)\\
философия как яйцо --- 3 части:\\
скорлупа = логика, белок = физика, желток = этика (самое вкусное).\\
Главные принципы неизменны:\\
а) космос разумен, совершенен, добродетелен\\
б) бог --- жизненный принцип, а не существо; при этом бога нельзя отделить от материи\\
бог во всём и всё есть Бог.\\
принцип пантеизма: $\exists$ план мироздания = судьба\\
Сенека: "судьба --- закон мироздания, управляющий разумом". рок=судьба=Бог. Эта судьба $\neq$ судьбе (в христианстве); в этом-то стоики и видели решение проблемы свободы. Принцип стоиков --- знать и подчиняться.

3. Эпикурейцы. Консервативны. Эпикур (341--279 гг). Философы сада\\
1) физика; 2) логика; 3) этика\\
Смысл философии --- в этике:\\
"Для счастья всегда есть время => для философии всегда есть время".\\
Мир Эпикура $\approx$ Мир демокрита, но 1 момент радикально всё меняет:\\
а) у демокрита двигаются по строго заданным траекториям;
б) у жпикура --- возможна случайность\\
Человек = атомы + пустота, как и весь мир\\
Нет дуализма души и тела! (с) Тит Лукреций Кат.\\
Нет никакого бессметрия души!\\
Мартин Хайдегер определяет смерть, как единственный реально наш поступок и у нас есть онтологический страх смерти. Эпикурс: "Приучая себя к ..."\\
Смерть --- есть лишение ощущения. \\
Теорема. $\exists$ мы <=> не $\exists$ смерть\\
Учение о наслаждении. Наслаждение --- отсутствие страданий.

4. Скептики знать ничего не можем ...
%%%%%%%%%%%%%%%%%%%%%%%%%%%%%%%%%%%%%%%%%%%%%%%%%%%%%%%%%%%%%%%%%%
\section{Лекция 6. 12 октября. Пропущена}
%%%%%%%%%%%%%%%%%%%%%%%%%%%%%%%%%%%%%%%%%%%%%%%%%%%%%%%%%%%%%%%%%%%
\section{Лекция 7. 19 октября}
Аристотель. Метафизика, науки: 1) теоритические; 2) практические; 3) продуктивные. Принцип=ради чего та или иная наука ищет знания.\\
а) практические --- поиск совершенства: этика и политика (всего 2)\\
б) теоритические --- знание ради знания (их 3):\\
1) 1ая философия (метафизика)\\
2) 2ая философия (физика)\\
3) математика (астрономия, гармоника, геометрия, но не искусство счета)\\
О философии: все другие науки ищут для чего-то, а философия --- ради знания.\\
\begin{tabular}{|c|c|c|}\hline
таблица &неподвижное &движущееся\\ \hline
сущее само по себе & метафизика & физика \\ \hline
сущее в ином & математика & ? \\ \hline
\end{tabular}\\
Предметы --- сущие по себе, а их качества --- сущие в ином.\\
Физика --- природа --- движение (апории зенона)\\
? --- случайное бытие, мб теорвер?

К сути метафизики: Андронник Радосский состявлял каталог работ Аристотеля, придумал слово "метафизика"=о бытии как таковом. Мета=после. Сведение многих оправданий метафизики в одно.\\
4 причины: 1)действующая; 2) материальная; 3)формальная; 4)целевая причина. Пример мраморной статуи.\\
Бытие --- единое (Элеаты);\\
$\exists$ не бытие (Платон);\\
всё, что не может быть бытием (Аристотель)\\
<<Платон мне друг, но истина дороже>> --- насчет бытия. Аргумент 3го человека:\\
идея человека <-> идея отношения \\
$\Updownarrow$ отношение $\nearrow$ \\
человек $\nearrow$

Мир идей разрастается до бесконечности. Смысл бытия:\\
1) как категории --- высшие роды значений бытия, 10 категорий: субстанция, качество, количество, отношение, действие, страдание, место, время, иметь и покоиться (последние две возможно придуманы не аристотелем).\\
субъект --- предикат S --- D; единичные термины --- субъекты, есть слова --- и то, и то; а есть термины, которые лишь D\\
2) акт и потенция (действительность и возможность)\\
материя := возможность вещи\\
форма вещи =: действительность вещи\\
форма $\nearrow$ => действительность $\nearrow$\\
глина ->  кирпичи -> дом \\
3) бытие акциденции (случайности, ?)\\
4) бытие как истина, небытие как ложь\\
$\forall$ вещь = синтез материи и формы\\
первоматерия = hyle = строй материал\\
индивид = соединение материи и формы \\
интелегибельность: форма --- суть вещи\\
Итог: бытие --- субстанция; субстанция все собственного смысла слова --- hyle; \\
собственное (во 2м значении) --- индивид\\
субстанция (собственное) --- форма по ???\\
Наконец, ум --- перводвигатель --- бог:
А <- B <- C <- ... либо $\infty$, либо $\exists$ первопричина = бог\\
Ум: чистая возможность => наверное $\exists$ чистая действительность =>  $\exists$ чистая форма =: бог\\
причина перводвигателя --- желания, любовь. Т.о. всё сущее стремится и любит Бога. У Аристотеля оптимистичнее картина космоса, чем у Платона.

О человеке:\\
1) учение о душе похоже на Платона, то есть. Аристотель: разумная, стремящаяся и растительная??\\
2) важен общественный статус "человек --- животное политическое (общественное)"; наука --- дело коллективное.\hrule

Средневековье. название относительно Европы. 476г. --- переворот Адаака. Китай? Африка? (римский император Рог). Философы и искусствоведы имеют другие границы средних веков: начало ~ 476г., конец ~ 15 век (книгопечатанье, падение Византии, Америка)\\
btw Ивану Федорову сожгли типографию\\
Острожь=Вильна <-- там первое издание русской библии + первая академия, в которой генерал ВСУ закончил магистратуру. Генуэские города в Крыму.\\
1453 г. --- турки берут Константинополь. В ХХ века выяснилось, что средневековое искусство и культура опирались на античность. Ж.ле Гофф --- 3 книги о средневековье

%%%%%%%%%%%%%%%%%%%%%%%%%%%%%%%%%%%%%%%%%%%%%%%%%%%%%%%%%%%%%%%%%%%
\section{Лекция 8. 26 октября}
( Соня (1))
Помочь челу! \\
Начало средних веков --- то время, когда в умы античных философов входят христианские идеи ~ 2 век н.э.\\
а) период аполлогетики, 2--3 век н.э.\\
1ая задача. Защита от критики античных традиционных школ (киников и тд). Как только апостол Павел перешел к новому завету, то ...\\
2ая задача борьбы с идеалогическими конкурентами (гностики, манихеи и т.д. --- религиозные философии). Апологеты: Татия, Афиногор, Феофил Антиохийский. Тертулиан: женщина не должна украшать тело. "Об одеяниях женщин".\\
б) период патристики --- философии отцов церкви, 4--7 век (запад, --10 век восток). Отец --- участник 1-х вселенских соборов (до разделения церквей). Это люди, которые формировали догматы Христа. Это: Василий Великий, Григорий Низкий, Иван Злот, Августин Аврелий.\\
в) период схоластики (только в западной философии), 13--15 века.\\
--- Это определенный тпи религиозной философии вокруг университетов. Схоластика = school философия. Обсуждали глупые вопросы: сколько ангелов поместится на острие иглы? может ли бог создать камень, который сам не сможет поднять? и т.д. Пьер Абеляр, Фома Аквинский, Альберт Абелий, Дунс Скотт, Ансельм Кентерберийский. Важно: евпропейские университеты перемещались по всей европе, универсальный язык --- латынь. \\
Основные принципы средневековой философии:\\
1) Теоцентризм --- фундаментальное разделение всего сущего на бытие несотворённое и бытие тварное. $\exists$ фундаментальная пропасть. Мы не поймем твари, не соотнеся с творцом. Всё псотигается через Бога.\\
2) Принцип Криационизма. Мир сотворен из ничего актом божественной воли! (свободной). Всё остальное обсуждаемо: например, что сначала форма или материя?\\
3) Принцип провинциализма --- всё в Мире происходит по воле творца. "Почему же люди творят плохое?" (ответ будет позже) \\
4) Ревеляционизм --- самые главные истины даны в откровении.

Юрисдикция --- судить студентку за загул в бар мог только университет (там была даже своя тюрьма). Игра университета на противоречиях между церковью и королем. Студенты досаждали граждан. Только где-то в 15 веке университетская бюрократия стала выполнять приказы папы.\\
Шаривари = кошачьи концерты = шумное безобразие, камни в окна, крики\\
Повод: если юная девушка выходила за старика (и наоборот)

О гендерных проблемах: потребительское отношение к женщине, на ней основная вина, уговорила Адама нарушить... <--- до 13 века. Но в конце 13 века --- феномен куртуазной любви. Это социально гендерная игра: поклонение рыцаря супруге своего сеньора. Лоцелот ухаживать пытался за женой артура. Артур доволен: чем больше поклонников у супруги, тем выше её статус. Такой стиль поведения задавал некий образец поведения (для элиты в первую очередь) => сдвиг взглядов европейцев. А дальше Ромео и Джульетта: молодой аболтус, уже побывавший во многих замужних местах, находит 13 летнюю девочку. Убивает её брата и т.д. Мораль: виноваты родители, которые не создали достаточных условий сексуальной жизни. \\
Важно: государство не при чем, не оно устраивало шаривари, а общество! Никакого тоталитаризма! Почему публичные дома были прямо в центре городов? --- Вырос возраст вступления в брак у мужчин на 4 года. Куда им идти? 1) соблазнять замужних; 2) молодых; 3) салон+\\
Свадьба --- тоже общественный контроль. Монастырь --- лучший способ избавиться от некрасивой жены. Культура авторитета и традиций. Средневековый чел хотел чувства опоры и уверенности (потому что чума, болезни, неурожаи, вонь) + шансы быть осужденным на вечные муки (после смерти) $\approx 100\%$. 
%%%%%%%%%%%%%%%%%%%%%%%%%%%%%%%%%%%%%%%%%%%%%%%%%%%%%%%%%%%%%%%%%%%
\section{Лекция 9. 2 ноября}
( Соня (2)). Татары первые заболели чумой, заразили генуэсцев\\
Средневековый чел ищет опуру в авторитете и традициях\\
Кутюма --- традиция, которую давно никто не оспаривал. "Исповедь блаженного Августина". Ссылки на авторитет $\nearrow$, в своих текстах изобретать --- безнравственно, козни дьявола\\
Цитировали: библию, отцов церкви, арабов (иногда просто выдумывали)\\
Середина средних веков: ссылки на античных авторов, "Надо отобрать у язычников их интеллектуальное наследие". Появляются т.н. глоссы (сборник изречений, откомментированных). \\
Доказательство чудом. "Чудеса Девы"\\
История Жанны Дарк (100 летняя война), ей <16 лет, король послушал её, напал и успех $\nearrow$. У жанны были видения и к ним прислушивались. Для средневекового чела природа --- только самое необычное; обыденное в природе --- не интересует. Средневековая культура --- культура символизма. Символ --- предмет узнавания, знак. Есть язык обычный, а есть язык символов. Церковь долго сопротивлялась переводу Библии. Иначе читать будут "Песнь песней", а там эротика одна. Драг.камни --- символы. Растения: гроздь винограда --- христос, проливающий кровь, розы,...., а вот яблоко --- символ зла.\\
Как измерить время, достаточное для служения богу? ---От бога ничего не скроешь!\\
Иохан Хэзинген (Осень средневековья)\\
Эпоха наивного бесстыдства, не знали категории детства (появилась лишь в 18 веке). "Ф.Арцес: Ребенок и детство при старом порядке". (старый порядок --- до великой французской революции). Эруар --- врач при французском дворе, при нем рос Людовиг. До 7 лет Людовигу разрешали разнузданно жить. 1987 год в СССР факультатив "Этика и психология семейной жизни"\\
О человеке (универсалии)\\
Номиналисты --- уверсалии $\exists$ независимо от вещей\\
реалисты --- уверсалии лишь общее название\\
Опять принцип теоцентризма. Человек --- царь природы (в античности звезды и планеты главнее человека)\\
В. Лосский "Догматическое богословие"\\
Но миссия не выполнима!\\
Концепция первородного греха, она менялась:\\
а) ранняя (Августина):
1) Порок. 2)Болезнь души (так как сопровождается вся физическая природа --- искаженная), люди --- уроды, мутанты.\\
3) Первородный грех наследуемый (это гинетическое заболевание)\\
NB. античность не знала философии истории, тогда не было вопросов смасла истории. Августин пытался понять, что означает захват Рима готами: пофиг с армией, флотом, Римом... А вот является ли жертва насильника грешной? Подмена проблемы (теории заговора --- признак)\\
Фома Аквинский (1225--1274 гг). Первородный грех не утратил способности, но потратил внешний природе; лишились сверхестесственного дара бога.

Тридентский собор 1545-1563 гг\\
(через 10 лет начнется научная революция)\\
Была принята концепция Ангсельма Кентерберийского и Иоанна Дунс Скотта: первородный грех --- лишь недостаток праведности, которую Адам не сумел сохранить. Этот недостаток умаляется крещением. Мы можем познать природу, мы не уроды. Стимул для развития науки --- реформация:\\
1) Учение о предопределении\\
2) Отсутствие посредника между богом и человеком\\
Лютер перевел Библию на Deutsch. Лютеране взяли из позднего Августина: Бог всех разделил на грешников и праведников. По какому-то признаку. Но мы-то не знаем, поэтому надо трудиться, а не молиться. Дух капиталлизма: Вебер "Протестантская этика и дух капиталлизма". США стала католической ток в ХХ веке.

%%%%%%%%%%%%%%%%%%%%%%%%%%%%%%%%%%%%%%%%%%%%%%%%%%%%%%%%%%%%%%%%%%%
\section{Лекция 10. 9 ноября. Пропущена} пропустил
%%%%%%%%%%%%%%%%%%%%%%%%%%%%%%%%%%%%%%%%%%%%%%%%%%%%%%%%%%%%%%%%%%%
\section{Лекция 11. 16 ноября. Предстоит}
( Соня (3)) предстоит
%%%%%%%%%%%%%%%%%%%%%%%%%%%%%%%%%%%%%%%%%%%%%%%%%%%%%%%%%%%%%%%%%%%
\section{Лекция 12. 23 ноября. Предстоит} предстоит
%%%%%%%%%%%%%%%%%%%%%%%%%%%%%%%%%%%%%%%%%%%%%%%%%%%%%%%%%%%%%%%%%%%
\section{Лекция 13. 30 ноября. Предстоит} предстоит
%%%%%%%%%%%%%%%%%%%%%%%%%%%%%%%%%%%%%%%%%%%%%%%%%%%%%%%%%%%%%%%%%%%
\section{Лекция 14. 7 декабря. Предстоит} предстоит
%%%%%%%%%%%%%%%%%%%%%%%%%%%%%%%%%%%%%%%%%%%%%%%%%%%%%%%%%%%%%%%%%%%
\section{Лекция 11. 14 декабря. Предстоит} предстоит

\section{Семинар с Ваней 28.09.22}
Свобода. Нужна ли вам свобода?\\
АС: 20 лет назад да, сейчас нет \\
Коля: свобода нужна ученым, не нужна дуракам --- свобода лишь инструмент\\
Коллективная не нужна, личная --- да\\
Сартр инквизитору: да, ты сделаешь людей счастливыми, но они перестанут быть людьми!\\
Коля: а почему бы тогда не стать цыпленком?\\
на листочках ответ на вопрос: почему согласно сартру невозможен абсолютный мезантроп? 
\end{document}
