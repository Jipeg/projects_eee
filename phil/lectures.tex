\documentclass[a4paper,12pt]{article}

\usepackage{cmap}					% поиск в PDF
\usepackage[T2A]{fontenc}			% кодировка
\usepackage[utf8]{inputenc}			% кодировка исходного текста
\usepackage[english,russian]{babel}	% локализация и переносы

\author{ничего не забыл и лекций не пропустил Николай Андронов}
\title{Конспект лекций Милославова А.С.}
\date{\today}

\begin{document}
\maketitle
\section{Лекция 1}
Ваш бюст и походка напомнили мне античность. Философия --- любовь к мудрости. Физики изучают нечто изменчивое. Главное в природе --- движение. Матан --- вечное, неизменное, существующие благодаря существуемым???.\\
3 типа мировоззрения: миф, религия, философия.

1. Миф: синкритичное т.е. слитное. этика+эстетика --- всё соединено. Лосев "Диалектика мифа" (1920е) эпизод из конца 19-го века. \\
Троянская война. 3 богини: Афродита, Гера и Афина решили выяснить, кто самая красивая. (боги такие же как люди, с теми же свойствами). Они выбрали Париса, чтобы он рассудил их. Афродита предложила Елену, жену Минелая. Минелай и все цари чувствовали коллективное унижение. Про Агамемнома и его дочь. Кризис родоплеменных отношений.\\
Итог: синкритизм, эмоции и чувства над разумом, коллективное над индивидуальным

2. Религия: удвоение Мира. Физиология верующего. Процедура обучения молитвенным слезам. Блаженная Ксения

3. Философия. Разум\\
Почему в греции появилась философия и наука? Комплекс причин:

1. Крито-мекенская культура, далее архейцы, далее классическая греция (гомер, полисы и т.д., греки понимали прошлое). ОДновременно: Спарта --- монархия, Афины --- демократия. \\
а) в греции царь = первый среди равных, а не бог --- фараон, жреческие функции\\
б) демократия: значительная часть населения участвует в управлении (но не все)\\
в) тиран --- правитель, не наследник\\
Есть способы разные => надо сравнивать => надо определить: справедливость, благо и т.д. - ?\\
Платон.государство: справедливое государство\\
Печаль древнего грека = изгнание из города => философия

2. Мореплавание, греки впитывают в себе моменты других культур, фалес привез геометрию из Египта\\
первые школы: не Афины! Милетус, Эфес, пифагорийцы еще где-то ??? \\
--- всё это след диалога культур.\\
корабельная ситуация: знания капитана определяло судьбу тех, кто на корабле\\
логос=слово капитана =(софисты) "великий властитель" = огонь (как стихия)

3. Древнегреческая религия.\\
а) не знали священных книг, текстов\\
б) олимпийские боги (официальная религия в классике) не являлись образцами в морально-этическом плане, их боялись, задобряли, но не уважали; Афина = псина, муха (с) Арес (Иллиада)\\
Ванхические девичники, Ванх --- Дионис.\\
в) вопрос о бессметрии души (отличие от Египта)\\
в египте: жизнь на земле --- подготовка к "аиду"\\
греки: после смерти --- ущербное состояние\\
Одиссея: попадает Одиссей в Аид и встречает душу Ахилеса, была назначена царем мертвых. "Я б лучше был батраком на земле, чем тут царем среди мертвых"\\
г) боги = очеловеченные силы природы\\
д) герои = сын бога и смертной женщины \\
смысл героизма = противостояние судьбе, року\\
принцип фатализма: рок, судьба диктуют как жить\\
греческий герой: судьба не мешает мне преследовать мои планы\\
е) child free и ЛГБТ\\
Эдип убил даже не зная что это был его pap\'a. А в городе чудище, треба тянок. Убил Эдип чудище, Эдипа царем сробили, женился не зная на своей mam\'a. Когда всё вскрылось, то все сдохли. Он пытался, он герой!\\
ж) Античная философия --- попытка понять гармонию и красоту космоса. Лосев "История эстетики". Никакого бога вне мироздания! Есть лишь космос, который был всегда.
\section{Лекция 2}
Мы остановились на Иллиадах.

Демокрит. Античный атомизм. Атом --- бытие Иллиады. Всё сущее = атомы, перемещающиеся в пустоте. Демокрита не изгнали, а наградили за его трактат "О природе". Парменид и другие от Фалеса до Демокрита --- всё "о природе и космосе". Дальше --- общество и человек!

Софист --- это эксперт, платный учитель красноречия. Они стали публично учить людей, просвещать. Именно они совершили переворот. Пример софиста --- Протагор: человек --- это мера всех вещей. Какой ветер: cold или hot? Всегда есть 2 мнения, а человек выбирает => человек --- мера. Предшественники экзистенциализма. Различают законы природы и законы общества. Нарушение природы => не ущерб; нарушение общества => ущерб, если поймают!

Сократ. знаем от Платона и Ксенофонта (не мог ничего придумать, но писал, когда сам был old) --- 2 разных Сократа. Сократ считал, что человек носит уже истину внутри себя и задача философа --- помочь людям родить мысль (как окушерка). Метод маевтики (наводящие вопросы). Но, чтобы мы захотели родить истину, надо разозлить людей. Жалил сограждан своими вопросами как овод скотину!\\
Апологея Сократа. Федон.\\
1. добродетели $\approx$ счастье, добро $\approx$ знание. Люди в тисках незнания. Человек = разумная душа. Сократ рулил евпропеоидной философией до середины 19го века. А потом стали работать гильятиной: кто мешает общественному балгу --- того в минус. Битва при Галиполе

Платон. Кодр спас город. Жизнь пришла в упадок! демократия --- плохо, така как убили Сократа. О Платоне --- Лосев "Плотон". \\
Как устроить государство?\\
1) теория идей; 2) бессмертие души; 3) знание как припоминание; 4) учение о государстве; 5) космология Алптона

1. Идея --- это не мысль, а то по поводу чего мысль думает! Мир идей --- над небом. Идея единого: всё $\exists$ в силу единства. Первый теоретик тоталитарного государства. Признаки тоталитарности:
1) $\exists$ группа лиц, претендующие на владение истины, управляют\\
2) ограничение на частную или личную собственность\\
2) вмешательство в личную жизнь государством\\
4) система психологического образования (пионеры)\\
Все это было у Платона. Вы --- для государства!

2. Учение о душе. Что есть в нас, чтобы познать мир идей? Органы чувств часто приводят нас в заблуждение, не подходит. Болезни не подходят. Войны ради денег, а они только для заботы о теле, тело --- "плохо"\\
=> $\exists$ душа --- бессмертная, нематериальная так как мир идей такой. 4 аргумента души. Душа --- повозка с двумя лошадьми: повозка=разум, лошади=аффектно-инстинктивная и вожделеющая части души\\
если разум справляется, то душа --- в мир идей; каждой части души --- добродетель:\\
аффективная --- умеренность --- люди (простолюдины)\\
вожделеющая --- мужество --- стражи\\
разум --- мудрость --- философы\\
Так вот ответ: душа уже была в Мире идей. Познание = припоминание.

\section{Лекция 5 октября}

\section{Семинар с Ваней 28.09.22}
Свобода. Нужна ли вам свобода?\\
АС: 20 лет назад да, сейчас нет \\
Коля: свобода нужна ученым, не нужна дуракам --- свобода лишь инструмент\\
Коллективная не нужна, личная --- да\\
Сартр инквизитору: да, ты сделаешь людей счастливыми, но они перестанут быть людьми!\\
Коля: а почему бы тогда не стать цыпленком?\\
на листочках ответ на вопрос: почему согласно сартру невозможен абсолютный мезантроп? 
\end{document}
