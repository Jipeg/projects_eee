\documentclass[a4paper,12pt]{article}

\usepackage{cmap}					% поиск в PDF
\usepackage[T2A]{fontenc}			% кодировка
\usepackage[utf8]{inputenc}			% кодировка исходного текста
\usepackage[english,russian]{babel}	% локализация и переносы

\author{Николай Андронов}
\title{Философские заметки}
\date{\today}

\begin{document}
\maketitle
\section{Как мораль делает нас слабыми}
Понятие морали тесно связано с разделением человеческих поступков на "плохие" и "хорошие", точнее на аморальные и моральные (возможно не прям хорошие, но нейтральные). Для жизни общества это очень важное разделение, потому что плохие поступки ведут общество к упадку и тд. Поэтому уже давно возникли целые системы морали такие, как религиозная мораль, светский гуманизм и тд. Как правило, каждый человек придерживается той или иной системы, а может даже сочетанием нескольких. Это позволяет дать простой ответ почему конкретный индивид не совершает аморальных поступков: "не укради, не убий и тд" или "светский гуманнизм" и тд. Наша цель показать, что такой ответ простой, но неправильный. 

Основная идея состоит в том, что мораль как таковая уже зашита в человека с рождения (потому что "передается по наследству" и развивается по законам биологической эволюции, то есть мораль в какой-то мере присуща и животным (ну а человеку как самому животному животному, конечно, больше всех) и тесно связана с чувством страха. В процессе эволюции сформировался некий рефлекс: если кого-то ударить, то скорее всего этот кто-то даст сдачи. Если взрослый лев ударит чьего-то детеныша-львенка, то ему прилет от родителей или сразу от всей стаи. Обратимся к примеру домашних животных: есть куча видео из тиктока, как кот что-то порвал или разбросал корм, а потом сидит-притворяется будто он не при чем (а всё потому, что уже знает, что в таких случаях хозяйка будет зла на него).

Рассмотрим пример чиновника, который решился украсть из бюджета денег и купить квартиру. Эту квартиру он не может зарегистрировать на себя, поэтому регистрирует на жену или любовницу. От этого чиновник становится более зависимым от нее, потому что она всегда может пригрозить всё рассказать полицаям. Заставит его ходить в магазин, выбрасывать мусор, дарить ей цветы каждый день, заствит бросить всех остальных любовниц... одним словом власть чиновника уменьшится, он станет более зависимым. Получается, что совершение аморальных поступков уменьшает человеческую власть, делает человека более слабым, добавляет страху. 

Можно сделать вывод, что чем больше чиновник ворует, тем больше у него есть "в запасе" власти. И если он по неаккуратности исчерпает лимит этой власти, то потеряет ее совсем --- уволят или арестуют. 

Аналогичная мысль есть во вселенной звездных войн. Когда Люк Скайуокер в порыве гнева хочет совершить аморальный поступок --- убить беззащитного императора, а потом чуть не убивает своего отца, Люк практически теряет власть и вот-вот будет подчиняться воле этого императора (станет его учеником). В звездных войнах такой переход считают возможным только со светлой на темную сторону. Однако он справедлив в обоих направлениях: если злодей вдруг совершит добрый поступок, то он не сможет так просто остановится, потому что перестанет выносить своих злодеев-товарищей, которые не стали совершать этот добрый поступок.

Поэтому правильный ответ на вопрос такой: мы не делаем
\end{document}
